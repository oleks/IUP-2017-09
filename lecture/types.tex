\begin{frame}

\frametitle{IEEE-754 Standard for Floating-Point Arithmetic?}

\begin{itemize}

\item Supported in hardware.

\end{itemize}

\end{frame}


\begin{frame}

\frametitle{Numerical Types You Might be Familiar With}

\begin{itemize}

\item[\textbf{R}] \hfill\\

\item[\texttt{numeric}] IEEE-754 Double Precision Floating-Point

\item[\textbf{Python}] \hfill\\

\item[\texttt{float}] IEEE-754 Double Precision Floating-Point

\item[\textbf{Matlab}] \hfill\\

\item[\texttt{single}] IEEE-754 Single Precision Floating-Point

\item[\texttt{double}] IEEE-754 Double Precision Floating-Point

\end{itemize}

\begin{center}

How did we get here? How does this IEEE-754 standard work?

\end{center}

\end{frame}
