\begin{frame}

\frametitle{Then Came Positional Notation}

\begin{center}

An approximation of the real numbers \textbf{\underline{to a desired degree of
accuracy}}.

\end{center}

\begin{itemize}

\item Historically, it greatly simplified arithmetic.

\item All integers, and many fractions can be represented exactly.

\item The rest is approximated to a desired degree of accuracy.

\end{itemize}

\end{frame}

\begin{frame}

\frametitle{Integers in Positional Notation}

\begin{itemize}

\item An integer is represented by a finite sequence of digits:

\[
d_n \cdots d_2 d_1 d_0
\]

Where $0 \leq d_i < b$, and $b$ is called a \emph{base} or \emph{radix}.

\item This has the following mathematical interpretation:

\[
\left\llbracket d_n \cdots d_0 \right\rrbracket_b =
d_n \cdot b^n + \cdots + d_2 \cdot b^2 + d_1 \cdot b^1 + d_0 \cdot b^0
\]

\item For instance, for the number $8125$, in base $10$:

\begin{align*}
\left\llbracket 8125 \right\rrbracket_{10} &=
8 \cdot 10^3 + 1 \cdot 10^2 + 2^\cdot 10^1 + 5 \cdot 10^0 \\
&= 8\cdot 1000 + 2 \cdot 100 + 1 \cdot 10 + 5 \cdot 1
\end{align*}

\end{itemize}

\end{frame}

\begin{frame}

\frametitle{Fractions in Positional Notation}

\begin{itemize}

\item The sequence may be interrupted by a radix point:

\[
\underbrace{d_n \cdots d_2 d_1 d_0}_{\text{(integral
part)}}~.~\underbrace{d_{-1} d_{-2} \cdots d_{-m}}_{\text{(fractional part)}}
\]

\item This has the following mathematical interpretation:\begin{align*}
\left\llbracket d_n \cdots d_{-m} \right\rrbracket_b =&~d_n \cdot b^n + \cdots
+ d_2 \cdot b^2 + d_1 \cdot b^1 + d_0 \cdot b^0 +\\
&~d_{-1} \cdot b^{-1} + d_{-2} \cdot b^{-2} + \cdots + d_{-m} \cdot b^{-m}
\end{align*}

\item For instance, the number $81.25$, in base $10$, has the following
mathematical interpretation:\begin{align*}
\left\llbracket 81.25 \right\rrbracket_{10} &=
8 \cdot 10^1 + 1 \cdot 10^0 + 2 \cdot 10^{-1} + 5 \cdot 10^{-2} \\
&= 8\cdot 10 + 1 \cdot 1 + 2 \cdot \frac{1}{10} + 5 \cdot \frac{1}{100}
\end{align*}

\end{itemize}

\end{frame}

\begin{frame}

\frametitle{Addition With Positional Notation}

\begin{itemize}

\item Place the sequences so they align along the radix point.

\item Pad with 0's if need be.

\item Add one digit at a time, left to right, carry $1$ if need be.

\end{itemize}

\vspace{\fill}

Examples:

\begin{minipage}{0.3\textwidth}
\begin{center}
{\setlength{\tabcolsep}{1pt}
\begin{tabular}{ccccc}
  & {\scriptsize $1$} \\
  & $3$ & $.$ & $7$ & $7$  \\
+ & $2$ & $.$ & $8$ & ${\color{gray}0}$   \\ \hline
  & $6$ & $.$ & $5$ & $7$
\end{tabular}
}
\end{center}
\end{minipage}%
\begin{minipage}{0.3\textwidth}
\begin{center}
{\setlength{\tabcolsep}{1pt}
\begin{tabular}{cccccc}
  & \\
  & $3$ & $7$ & $.$ & $7$ & ${\color{gray}0}$  \\
+ & ${\color{gray}0}$ & $0$ & $.$ & $2$ & $8$   \\ \hline
  & $3$ & $7$ & $.$ & $9$ & $8$
\end{tabular}
}
\end{center}
\end{minipage}%
\begin{minipage}{0.3\textwidth}
\begin{center}
{\setlength{\tabcolsep}{1pt}
\begin{tabular}{cccccc}
  & {\scriptsize $1$} \\
  & ${\color{gray}0}$ & $3$ & $.$ & $7$ & $7$  \\
+ & $2$ & $8$ & $.$ & ${\color{gray}0}$ & ${\color{gray}0}$ \\ \hline
  & $3$ & $1$ & $.$ & $7$ & $7$
\end{tabular}
}
\end{center}
\end{minipage}

\end{frame}

\begin{frame}

\frametitle{Multiplication With Positional Notation}

\vspace{\fill}

\begin{minipage}{0.7\textwidth}

\begin{itemize}

\item Align the sequences along \\ their right-most digit.

\item Disregard their radix points.

\item Multiply each digit of the top \\ by each digit of the bottom.

\item Align and add the products.

\item The number of decimal places \\ of the product is equal to the \\ sum of
the number of decimal \\ places in the operands.

\end{itemize}

\end{minipage}%
\begin{minipage}[t]{0.3\textwidth}

{\setlength{\tabcolsep}{1pt}
\begin{tabular}{ccccccc}
&          && $3$ & $.$ & $7$ & $7$  \\
&$\times$  && &   $2$ & $.$ & $8$  \\ \hline
&{\scriptsize $7\cdot 8$} && & & $5$ & $6$ \\
&{\scriptsize $7\cdot 8$} && & $5$ & $6$ & ${\color{gray}0}$ \\
&{\scriptsize $3\cdot 8$} && $2$ & $4$ & ${\color{gray}0}$ & ${\color{gray}0}$ \\
&{\scriptsize $7\cdot 2$} && & $1$ & $4$ & ${\color{gray}0}$ \\\
&{\scriptsize $7\cdot 2$} && $1$ & $4$ & ${\color{gray}0}$ & ${\color{gray}0}$ \\
$+$ &{\scriptsize $3\cdot 2$} && $6$ & ${\color{gray}0}$ & ${\color{gray}0}$ & ${\color{gray}0}$ \\
\hline
&& $1$ & $0.$ & $5$ & $5$ & $6$
\end{tabular}
}

\end{minipage}

\vspace{\fill}

\end{frame}
