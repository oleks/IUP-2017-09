\begin{frame}

\frametitle{Then Came Positional Notation}

\begin{center}

An approximation of the real numbers \textbf{\underline{to a desired degree of
accuracy}}.

\end{center}

\begin{itemize}

\item Historically, it greatly simplified arithmetic.

\item All integers, and many fractions can be represented exactly.

\item The rest is approximated to a desired degree of accuracy.

\end{itemize}

\end{frame}

\begin{frame}

\frametitle{Integers in Positional Notation}

\begin{itemize}

\item An integer is represented by a finite sequence of digits:

\[
d_n \cdots d_2 d_1 d_0
\]

Where $0 \leq d_i < b$, and $b$ is called a \emph{base} or \emph{radix}.

\item This has the following mathematical interpretation:

\[
\left\llbracket d_n \cdots d_0 \right\rrbracket_b =
d_n \cdot b^n + \cdots + d_2 \cdot b^2 + d_1 \cdot b^1 + d_0 \cdot b^0
\]

\item For instance, for the number $8125$, in base $10$:

\begin{align*}
\left\llbracket 8125 \right\rrbracket_{10} &=
8 \cdot 10^3 + 1 \cdot 10^2 + 2^\cdot 10^1 + 5 \cdot 10^0 \\
&= 8\cdot 1000 + 2 \cdot 100 + 1 \cdot 10 + 5 \cdot 1
\end{align*}

\end{itemize}

\end{frame}

\begin{frame}

\frametitle{Fractions in Positional Notation}

\begin{itemize}

\item The sequence may be interrupted by a radix point:

\[
\underbrace{d_n \cdots d_2 d_1 d_0}_{\text{(integral
part)}}~.~\underbrace{d_{-1} d_{-2} \cdots d_{-m}}_{\text{(fractional part)}}
\]

\item This has the following mathematical interpretation:\begin{align*}
\left\llbracket d_n \cdots d_{-m} \right\rrbracket_b =&~d_n \cdot b^n + \cdots
+ d_2 \cdot b^2 + d_1 \cdot b^1 + d_0 \cdot b^0 +\\
&~d_{-1} \cdot b^{-1} + d_{-2} \cdot b^{-2} + \cdots + d_{-m} \cdot b^{-m}
\end{align*}

\item For instance, the number $81.25$, in base $10$, has the following
mathematical interpretation:\begin{align*}
\left\llbracket 81.25 \right\rrbracket_{10} &=
8 \cdot 10^1 + 1 \cdot 10^0 + 2 \cdot 10^{-1} + 5 \cdot 10^{-2} \\
&= 8\cdot 10 + 1 \cdot 1 + 2 \cdot \frac{1}{10} + 5 \cdot \frac{1}{100}
\end{align*}

\end{itemize}

\end{frame}

\begin{frame}

\frametitle{Addition With Positional Notation}

\begin{itemize}

\item Place the sequences so they align along the radix point.

\item Pad with 0's if need be.

\item Add one digit at a time, left to right, with carry.

\end{itemize}

\vspace{\fill}

Examples:

\begin{minipage}{0.3\textwidth}
\begin{center}
{\setlength{\tabcolsep}{1pt}
\begin{tabular}{ccccc}
  & $.$ \\
  & $3$ & $.$ & $7$ & $7$  \\
+ & $2$ & $.$ & $8$ & ${\color{gray}0}$   \\ \hline
  & $6$ & $.$ & $5$ & $7$
\end{tabular}
}
\end{center}
\end{minipage}%
\begin{minipage}{0.3\textwidth}
\begin{center}
{\setlength{\tabcolsep}{1pt}
\begin{tabular}{cccccc}
  & \\
  & $3$ & $7$ & $.$ & $7$ & ${\color{gray}0}$  \\
+ & ${\color{gray}0}$ & $0$ & $.$ & $2$ & $8$   \\ \hline
  & $3$ & $7$ & $.$ & $9$ & $8$
\end{tabular}
}
\end{center}
\end{minipage}%
\begin{minipage}{0.3\textwidth}
\begin{center}
{\setlength{\tabcolsep}{1pt}
\begin{tabular}{cccccc}
  & $.$ \\
  & ${\color{gray}0}$ & $3$ & $.$ & $7$ & $7$  \\
+ & $2$ & $8$ & $.$ & ${\color{gray}0}$ & ${\color{gray}0}$ \\ \hline
  & $3$ & $1$ & $.$ & $7$ & $7$
\end{tabular}
}
\end{center}
\end{minipage}

\end{frame}

\begin{frame}

\frametitle{Doing Arithmetic In Positional Notation}

\begin{minipage}{0.5\textwidth}

\begin{center}
\textbf{Addition}
\end{center}

\begin{itemize}

\item Place the sequences so they align along the radix point.

\item Pad with 0's if need be.

\item Add one digit at a time, left to right, with carry.

\end{itemize}

\end{minipage}%
\begin{minipage}{0.5\textwidth}

\begin{center}
\textbf{Multiplication}
\end{center}

\begin{itemize}

\item Align the sequences along their right-most digit.

\item Disregard the radix point.

\item Multiply each digit of the top by each digit of the bottom.

\item Align and add the products.

\end{itemize}

\end{minipage}

\begin{center}

These algorithms work equally well in all bases.

\end{center}

\end{frame}


\begin{frame}

\frametitle{Properties}

\vspace{\fill}

\begin{center}

\begin{minipage}{0.5\textwidth}
\begin{center}
\textbf{Commutativity} \begin{align*}
a \oplus b &= b \oplus a \\
a \odot b &= b \odot a
\end{align*}
\end{center}
\end{minipage}%
\begin{minipage}{0.5\textwidth}
\begin{center}
\textbf{Associativity}~\frownie \begin{align*}
\p{a \oplus b} \oplus c &\neq a \oplus \p{b \oplus c} \\
\p{a \odot b} \odot c &\neq a \odot \p{b \odot c}
\end{align*}
\end{center}
\end{minipage}

\vspace{\fill}

\textbf{Distributivity}~\frownie \begin{align*}
a \odot \p{b \oplus c} &\neq a \odot b \oplus a \odot c
\end{align*}

\end{center}

\vspace{\fill}

\end{frame}
