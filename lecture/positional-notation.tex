\begin{frame}

\frametitle{Then Came Positional Notation}

\begin{center}

An approximation of the real numbers \textbf{\underline{to a desired degree of
accuracy}}.

\end{center}

\begin{itemize}

\item Historically, it greatly simplified arithmetic.

\item All integers, and many fractions can be represented exactly.

\item The rest is approximated to a desired degree of accuracy.

\end{itemize}

\end{frame}

\begin{frame}

\frametitle{Integers in Positional Notation}

\begin{itemize}

\item An integer is represented by a finite sequence of digits:

\[
d_n \cdots d_1 d_0
\]

Where $0 \leq d_i < b$, and $b$ is called a \emph{base} or \emph{radix}.

\item This has the following mathematical interpretation:

\[
\left\llbracket d_n \cdots d_1 d_0 \right\rrbracket =
d_n \cdot b^n + \cdots + d_1 \cdot b^1 + d_0 \cdot b^0
\]

\item For instance, the number $8125$, in base $10$, has the following
mathematical interpretation:

\begin{align*}
\left\llbracket 8125 \right\rrbracket_{10} &=
8 \cdot 10^3 + 1 \cdot 10^2 + 2^\cdot 10^1 + 5 \cdot 10^0 \\
&= 8\cdot 1000 + 2 \cdot 100 + 1 \cdot 10 + 5 \cdot 1
\end{align*}

\end{itemize}

\end{frame}

\begin{frame}

\frametitle{Fractions in Positional Notation}

\begin{itemize}

\item The sequence may be interrupted by a radix point:

\[
\underbrace{d_n \cdots d_1 d_0}_{\text{(integral part)}}~.~\underbrace{d_{-1}
d_{-2} \cdots d_{-m}}_{\text{(fractional part)}}
\]

\item This has the following mathematical interpretation:\begin{align*}
\left\llbracket d_n \cdots d_{-m} \right\rrbracket =&~d_n \cdot b^n + \cdots +
d_1 \cdot b^1 + d_0 \cdot b^0 +\\
&~d_{-1} \cdot b^{-1} + d_{-2} \cdot b^{-2} + \cdots + d_{-m} \cdot b^{-m}
\end{align*}

\item For instance, the number $81.25$, in base $10$, has the following
mathematical interpretation:\begin{align*}
\left\llbracket 81.25 \right\rrbracket_{10} &=
8 \cdot 10^1 + 1 \cdot 10^0 + 2^\cdot 10^{-1} + 5 \cdot 10^{-2} \\
&= 8\cdot 10 + 2 \cdot 1 + 1 \cdot \frac{1}{10} + 5 \cdot \frac{1}{100}
\end{align*}

\end{itemize}

\end{frame}

\begin{frame}

\frametitle{An Example of Positional Notation}

\begin{itemize}

\item The value $8.125$ is composed of 4 digits: $8$, $1$, $2$, $5$, separated
by a decimal point between $8$ and $1$.

\item The order of the digits, relative to the decimal point is significant,
and signifies a quantifier of a power of the base of the notation.

\item $8.125 = 8\cdot 1 + 1 \cdot 0.1 + 2 \cdot 0.01 + 5 \cdot 0.001$.

\item Some properties of real numbers do not hold for the approximation.

\end{itemize}

\begin{center}

\begin{tabular}{cccc}
$8$ & $1$ & $2$ & $5$ \\ \hline
$10^1$ & $10^{-1}$ & $10^{-2}$ & $10^{-3}$
\end{tabular}

\end{center}

\end{frame}

\begin{frame}

\frametitle{Doing Arithmetic In Positional Notation}

\begin{center}

These algorithms work equally well in all bases.

\end{center}

\end{frame}
