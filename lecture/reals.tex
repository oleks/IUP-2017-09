\begin{frame}

\frametitle{At First, There Were the Real Numbers}

\vspace{\fill}

\textbf{Examples:}

\begin{itemize}

\item Integers: $1$, $2$, $3$, $-5$, $42$, $-1337$, etc.

\item Fractions: $\frac{1}{2}$, $\frac{2}{3}$, $-\frac{3}{4}$, $\frac{5}{10}$,
$-\frac{125}{37}$, etc.

\item Irrational numbers: $\sqrt{2}$, $\pi$, $e$.

\end{itemize}

\textbf{Counter-Examples:}

\begin{itemize}

\item Imaginary numbers: $\sqrt{-1}$

\end{itemize}

\end{frame}

\begin{frame}

\frametitle{The Real Numbers Had Some Nice Properties}

\vspace{\fill}

\begin{center}

\begin{minipage}{0.5\textwidth}
\begin{center}
\textbf{Commutativity} \begin{align*}
a + b &= b + a \\
a \cdot b &= b \cdot a
\end{align*}
\end{center}
\end{minipage}%
\begin{minipage}{0.5\textwidth}
\begin{center}
\textbf{Associativity} \begin{align*}
\p{a + b} + c &= a + \p{b + c} \\
\p{a \cdot b} \cdot c &= a \cdot \p{b \cdot c}
\end{align*}
\end{center}
\end{minipage}

\vspace{\fill}

\textbf{Distributivity} \begin{align*}
a \cdot \p{b + c} &= a \cdot b + a \cdot c
\end{align*}

\end{center}

\vspace{\fill}

\end{frame}
