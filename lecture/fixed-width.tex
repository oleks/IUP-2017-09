\begin{frame}

\frametitle{Then Comes Fixed-Width Positional Notation}

\begin{itemize}

\item There is a fixed number of positions, $n$.

\item The result of each operation must fit in $n$ digits.

\end{itemize}

\vspace{\fill}

Example:

\begin{itemize}

\item Consider a 3-digit representation.

\item How would you represent $8.555$?

\item How would you represent $-8.555$?

\end{itemize}

\end{frame}

\begin{frame}

\frametitle{Rounding Modes}

Rounding modes available in IEEE-754:

\begin{itemize}

\item Round to nearest, break ties to even.

\item Round to nearest, break ties to odd.

\item Round towards $0$.

\item Round towards $+\infty$.

\item Round towards $-\infty$.

\end{itemize}

How do they pan out in practice?

\begin{itemize}

\item Round to nearest, with ties to even is usually default.

\item A program may choose its rounding mode, \\ and change while it is
running.

\item On GPUs, each floating-point instruction may be parametrized by the
desired rounding mode.

\end{itemize}

\end{frame}
